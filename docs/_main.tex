% Options for packages loaded elsewhere
\PassOptionsToPackage{unicode}{hyperref}
\PassOptionsToPackage{hyphens}{url}
%
\documentclass[
]{book}
\usepackage{amsmath,amssymb}
\usepackage{lmodern}
\usepackage{iftex}
\ifPDFTeX
  \usepackage[T1]{fontenc}
  \usepackage[utf8]{inputenc}
  \usepackage{textcomp} % provide euro and other symbols
\else % if luatex or xetex
  \usepackage{unicode-math}
  \defaultfontfeatures{Scale=MatchLowercase}
  \defaultfontfeatures[\rmfamily]{Ligatures=TeX,Scale=1}
\fi
% Use upquote if available, for straight quotes in verbatim environments
\IfFileExists{upquote.sty}{\usepackage{upquote}}{}
\IfFileExists{microtype.sty}{% use microtype if available
  \usepackage[]{microtype}
  \UseMicrotypeSet[protrusion]{basicmath} % disable protrusion for tt fonts
}{}
\makeatletter
\@ifundefined{KOMAClassName}{% if non-KOMA class
  \IfFileExists{parskip.sty}{%
    \usepackage{parskip}
  }{% else
    \setlength{\parindent}{0pt}
    \setlength{\parskip}{6pt plus 2pt minus 1pt}}
}{% if KOMA class
  \KOMAoptions{parskip=half}}
\makeatother
\usepackage{xcolor}
\usepackage{color}
\usepackage{fancyvrb}
\newcommand{\VerbBar}{|}
\newcommand{\VERB}{\Verb[commandchars=\\\{\}]}
\DefineVerbatimEnvironment{Highlighting}{Verbatim}{commandchars=\\\{\}}
% Add ',fontsize=\small' for more characters per line
\usepackage{framed}
\definecolor{shadecolor}{RGB}{248,248,248}
\newenvironment{Shaded}{\begin{snugshade}}{\end{snugshade}}
\newcommand{\AlertTok}[1]{\textcolor[rgb]{0.94,0.16,0.16}{#1}}
\newcommand{\AnnotationTok}[1]{\textcolor[rgb]{0.56,0.35,0.01}{\textbf{\textit{#1}}}}
\newcommand{\AttributeTok}[1]{\textcolor[rgb]{0.77,0.63,0.00}{#1}}
\newcommand{\BaseNTok}[1]{\textcolor[rgb]{0.00,0.00,0.81}{#1}}
\newcommand{\BuiltInTok}[1]{#1}
\newcommand{\CharTok}[1]{\textcolor[rgb]{0.31,0.60,0.02}{#1}}
\newcommand{\CommentTok}[1]{\textcolor[rgb]{0.56,0.35,0.01}{\textit{#1}}}
\newcommand{\CommentVarTok}[1]{\textcolor[rgb]{0.56,0.35,0.01}{\textbf{\textit{#1}}}}
\newcommand{\ConstantTok}[1]{\textcolor[rgb]{0.00,0.00,0.00}{#1}}
\newcommand{\ControlFlowTok}[1]{\textcolor[rgb]{0.13,0.29,0.53}{\textbf{#1}}}
\newcommand{\DataTypeTok}[1]{\textcolor[rgb]{0.13,0.29,0.53}{#1}}
\newcommand{\DecValTok}[1]{\textcolor[rgb]{0.00,0.00,0.81}{#1}}
\newcommand{\DocumentationTok}[1]{\textcolor[rgb]{0.56,0.35,0.01}{\textbf{\textit{#1}}}}
\newcommand{\ErrorTok}[1]{\textcolor[rgb]{0.64,0.00,0.00}{\textbf{#1}}}
\newcommand{\ExtensionTok}[1]{#1}
\newcommand{\FloatTok}[1]{\textcolor[rgb]{0.00,0.00,0.81}{#1}}
\newcommand{\FunctionTok}[1]{\textcolor[rgb]{0.00,0.00,0.00}{#1}}
\newcommand{\ImportTok}[1]{#1}
\newcommand{\InformationTok}[1]{\textcolor[rgb]{0.56,0.35,0.01}{\textbf{\textit{#1}}}}
\newcommand{\KeywordTok}[1]{\textcolor[rgb]{0.13,0.29,0.53}{\textbf{#1}}}
\newcommand{\NormalTok}[1]{#1}
\newcommand{\OperatorTok}[1]{\textcolor[rgb]{0.81,0.36,0.00}{\textbf{#1}}}
\newcommand{\OtherTok}[1]{\textcolor[rgb]{0.56,0.35,0.01}{#1}}
\newcommand{\PreprocessorTok}[1]{\textcolor[rgb]{0.56,0.35,0.01}{\textit{#1}}}
\newcommand{\RegionMarkerTok}[1]{#1}
\newcommand{\SpecialCharTok}[1]{\textcolor[rgb]{0.00,0.00,0.00}{#1}}
\newcommand{\SpecialStringTok}[1]{\textcolor[rgb]{0.31,0.60,0.02}{#1}}
\newcommand{\StringTok}[1]{\textcolor[rgb]{0.31,0.60,0.02}{#1}}
\newcommand{\VariableTok}[1]{\textcolor[rgb]{0.00,0.00,0.00}{#1}}
\newcommand{\VerbatimStringTok}[1]{\textcolor[rgb]{0.31,0.60,0.02}{#1}}
\newcommand{\WarningTok}[1]{\textcolor[rgb]{0.56,0.35,0.01}{\textbf{\textit{#1}}}}
\usepackage{longtable,booktabs,array}
\usepackage{calc} % for calculating minipage widths
% Correct order of tables after \paragraph or \subparagraph
\usepackage{etoolbox}
\makeatletter
\patchcmd\longtable{\par}{\if@noskipsec\mbox{}\fi\par}{}{}
\makeatother
% Allow footnotes in longtable head/foot
\IfFileExists{footnotehyper.sty}{\usepackage{footnotehyper}}{\usepackage{footnote}}
\makesavenoteenv{longtable}
\usepackage{graphicx}
\makeatletter
\def\maxwidth{\ifdim\Gin@nat@width>\linewidth\linewidth\else\Gin@nat@width\fi}
\def\maxheight{\ifdim\Gin@nat@height>\textheight\textheight\else\Gin@nat@height\fi}
\makeatother
% Scale images if necessary, so that they will not overflow the page
% margins by default, and it is still possible to overwrite the defaults
% using explicit options in \includegraphics[width, height, ...]{}
\setkeys{Gin}{width=\maxwidth,height=\maxheight,keepaspectratio}
% Set default figure placement to htbp
\makeatletter
\def\fps@figure{htbp}
\makeatother
\setlength{\emergencystretch}{3em} % prevent overfull lines
\providecommand{\tightlist}{%
  \setlength{\itemsep}{0pt}\setlength{\parskip}{0pt}}
\setcounter{secnumdepth}{5}
\usepackage{booktabs}
\ifLuaTeX
  \usepackage{selnolig}  % disable illegal ligatures
\fi
\usepackage[]{natbib}
\bibliographystyle{plainnat}
\IfFileExists{bookmark.sty}{\usepackage{bookmark}}{\usepackage{hyperref}}
\IfFileExists{xurl.sty}{\usepackage{xurl}}{} % add URL line breaks if available
\urlstyle{same} % disable monospaced font for URLs
\hypersetup{
  pdftitle={Estadística I \& Estadística Descriptiva},
  pdfauthor={Coordinación de Estadística - UFME},
  hidelinks,
  pdfcreator={LaTeX via pandoc}}

\title{Estadística I \& Estadística Descriptiva}
\author{Coordinación de Estadística - UFME}
\date{}

\usepackage{amsthm}
\newtheorem{theorem}{Theorem}[chapter]
\newtheorem{lemma}{Lemma}[chapter]
\newtheorem{corollary}{Corollary}[chapter]
\newtheorem{proposition}{Proposition}[chapter]
\newtheorem{conjecture}{Conjecture}[chapter]
\theoremstyle{definition}
\newtheorem{definition}{Definition}[chapter]
\theoremstyle{definition}
\newtheorem{example}{Example}[chapter]
\theoremstyle{definition}
\newtheorem{exercise}{Exercise}[chapter]
\theoremstyle{definition}
\newtheorem{hypothesis}{Hypothesis}[chapter]
\theoremstyle{remark}
\newtheorem*{remark}{Remark}
\newtheorem*{solution}{Solution}
\begin{document}
\maketitle

{
\setcounter{tocdepth}{1}
\tableofcontents
}
\hypertarget{presentaciuxf3n-del-curso}{%
\chapter*{Presentación del curso}\label{presentaciuxf3n-del-curso}}
\addcontentsline{toc}{chapter}{Presentación del curso}

This is a \emph{sample} book written in \textbf{Markdown}. You can use anything that Pandoc's Markdown supports; for example, a math equation \(a^2 + b^2 = c^2\).

\hypertarget{tuxf3picos-buxe1sicos-de-estaduxedstica}{%
\chapter{Tópicos básicos de estadística}\label{tuxf3picos-buxe1sicos-de-estaduxedstica}}

This is a \emph{sample} book written in \textbf{Markdown}. You can use anything that Pandoc's Markdown supports; for example, a math equation \(a^2 + b^2 = c^2\).

\hypertarget{conceptos}{%
\section{Conceptos}\label{conceptos}}

\hypertarget{datos}{%
\subsection{Datos}\label{datos}}

Un dato es cualquier evento o hecho que no ha sido dotado de significado, es decir, es cualquier hecho del cual no se puede dar interpretación alguna.

Un ejemplo de este concepto, es cuando tratamos de responder la pregunta ¿por qué nos detenemos al caminar, cuando encontramos un semáforo en rojo? ¿Cuál es el dato? ¿Cuál es el significado?

\hypertarget{section}{%
\subsection{}\label{section}}

\hypertarget{gruxe1ficos-descriptivos}{%
\section{Gráficos descriptivos}\label{gruxe1ficos-descriptivos}}

\hypertarget{tuxf3picos-buxe1sicos-de-estaduxedstica-1}{%
\chapter{Tópicos básicos de estadística}\label{tuxf3picos-buxe1sicos-de-estaduxedstica-1}}

\hypertarget{conceptos-1}{%
\section{Conceptos}\label{conceptos-1}}

\hypertarget{datos-1}{%
\subsection*{Datos}\label{datos-1}}
\addcontentsline{toc}{subsection}{Datos}

Un dato es cualquier evento o hecho que no ha sido dotado de significado, es decir, es cualquier hecho del cual no se puede dar interpretación alguna.

Un ejemplo de este concepto, es cuando tratamos de responder la pregunta ¿por qué nos detenemos al caminar, cuando encontramos un semáforo en rojo? ¿Cuál es el dato? ¿Cuál es el significado?

\hypertarget{informaciuxf3n}{%
\subsection*{Información}\label{informaciuxf3n}}
\addcontentsline{toc}{subsection}{Información}

\textbf{Información = Datos + Significado}

Los datos existen independiente de quien observa, y cuando una persona adquiere datos y los dota de significado, estos se convierten en información. Otra forma de entenderlo es:

\textbf{Información = Datos + Reglas para decodificar}

En el ejemplo anterior, el decodificador es la persona que va caminando, y el significado (reglas para decodificar) que le damos al semáforo al estar en rojo, viene de las reglas sociales que indican como actuar en determinadas situaciones.

\textbf{En estadística, mediante el uso de distintas herramientas, dotaremos de significado a los datos, para así generar información de utilidad en distintos fenómenos de estudio propios de su área.}

\hypertarget{tipos-de-variables}{%
\subsection*{Tipos de variables}\label{tipos-de-variables}}
\addcontentsline{toc}{subsection}{Tipos de variables}

Otro concepto básico de estadística, es el tipo de variable, es decir, el tipo de dato que estoy observando. La clasificación es la siguiente:

\begin{itemize}
\tightlist
\item
  Cualitativas (Nominales y Ordinales): variables no numéricas que pueden o no llevar un orden, respectivamente.
\item
  Cuantitativas (Discretas y Continuas): variables numéricas que pueden o no ser enumeradas, respectivamente.
\end{itemize}

\textbf{Ejemplo}: Determinar la clasificación de las siguientes variables: tiempo, dinero, altura, cantidad de vecinos en el lugar donde vivo, grado de conformidad (conforme, medianamente conforme, nada conforme) respecto a un servicio, color de pelo de un grupo de personas.

\hypertarget{poblaciuxf3n-y-muestra}{%
\subsection*{Población y Muestra}\label{poblaciuxf3n-y-muestra}}
\addcontentsline{toc}{subsection}{Población y Muestra}

\begin{itemize}
\tightlist
\item
  \textbf{Población}: La población es el conjunto de todos los sujetos de interés en un estudio.
\item
  \textbf{Muestra}: La muestra es un subconjunto de la población a través de los cuales el estudio recoge los datos. Aspectos importantes de la muestra son el tamaño y distribución de las características.
\end{itemize}

Determinar en cada caso la población y la muestra:

\begin{itemize}
\tightlist
\item
  Se realiza un sondeo para determinar los rubros con mayor inflación de venta de mercado en Santiago, para ello se estudia el rubro con mayor ingreso líquido de ventas, en algunas de las comunas de Santiago.
\item
  La encuesta ENUSC elabora anualmente un informe respecto a la seguridad ciudadana, para ello, se contacta a una cantidad de personas determinadas de cada región del país, dando así, resultados a nivel nacional y regional.
\end{itemize}

\hypertarget{paruxe1metros-y-estaduxedsticos}{%
\subsection*{Parámetros y Estadísticos}\label{paruxe1metros-y-estaduxedsticos}}
\addcontentsline{toc}{subsection}{Parámetros y Estadísticos}

Ambos conceptos son utilizados de manera frecuente en distintos medios de comunicación, cometiéndose el error de tratarlos como sinónimos. Sin embargo, tienen definiciones totalmente distintas:

\begin{itemize}
\tightlist
\item
  \textbf{Parámetros}: característica numérica de resumen de la población.
\item
  \textbf{Estadísticos}: característica numérica de resumen de la muestra.
\end{itemize}

Veamos el siguiente gráfico y, distingamos el parámetro y estadístico correspondiente.

\begin{center}\includegraphics{_main_files/figure-latex/unnamed-chunk-2-1} \end{center}

\hypertarget{estimador-y-estimaciuxf3n}{%
\subsection{Estimador y Estimación}\label{estimador-y-estimaciuxf3n}}

\begin{itemize}
\tightlist
\item
  \textbf{Estimador}: Un estimador es un estadístico usado para aproximar (incertidumbre) el valor de un parámetro. Usualmente no cambia la técnica entre la población y la muestra, por ejemplo, si deseo aproximar la proporción de bolitas rojas en la población, se usaría la proporción de bolitas rojas en la muestra.
\item
  \textbf{Estimación}: Una estimación es el número que resulta de aplicar el estimador a una muestra particular. Esto difiera levemente de la definición anterior, ya que en términos estrictos, el estimador solo es la ``fórmula'', y la estimación es el valor resultante al aplicar la fórmula. Sin embargo, hoy en día es muy común encontrar textos en donde el estimador se considera tanto para la fórmula como para el valor obtenido.
\end{itemize}

\begin{center}\includegraphics{_main_files/figure-latex/unnamed-chunk-3-1} \end{center}

¿Cuándo diríamos que una estimación es buena?

\hypertarget{variabilidad-muestral}{%
\subsection{Variabilidad muestral}\label{variabilidad-muestral}}

\hypertarget{gruxe1ficos-descriptivos-1}{%
\section{Gráficos descriptivos}\label{gruxe1ficos-descriptivos-1}}

\hypertarget{parts}{%
\chapter{Parts}\label{parts}}

You can add parts to organize one or more book chapters together. Parts can be inserted at the top of an .Rmd file, before the first-level chapter heading in that same file.

Add a numbered part: \texttt{\#\ (PART)\ Act\ one\ \{-\}} (followed by \texttt{\#\ A\ chapter})

Add an unnumbered part: \texttt{\#\ (PART\textbackslash{}*)\ Act\ one\ \{-\}} (followed by \texttt{\#\ A\ chapter})

Add an appendix as a special kind of un-numbered part: \texttt{\#\ (APPENDIX)\ Other\ stuff\ \{-\}} (followed by \texttt{\#\ A\ chapter}). Chapters in an appendix are prepended with letters instead of numbers.

\hypertarget{footnotes-and-citations}{%
\chapter{Footnotes and citations}\label{footnotes-and-citations}}

\hypertarget{footnotes}{%
\section{Footnotes}\label{footnotes}}

Footnotes are put inside the square brackets after a caret \texttt{\^{}{[}{]}}. Like this one \footnote{This is a footnote.}.

\hypertarget{citations}{%
\section{Citations}\label{citations}}

Reference items in your bibliography file(s) using \texttt{@key}.

For example, we are using the \textbf{bookdown} package \citep{R-bookdown} (check out the last code chunk in index.Rmd to see how this citation key was added) in this sample book, which was built on top of R Markdown and \textbf{knitr} \citep{xie2015} (this citation was added manually in an external file book.bib).
Note that the \texttt{.bib} files need to be listed in the index.Rmd with the YAML \texttt{bibliography} key.

The RStudio Visual Markdown Editor can also make it easier to insert citations: \url{https://rstudio.github.io/visual-markdown-editing/\#/citations}

\hypertarget{blocks}{%
\chapter{Blocks}\label{blocks}}

\hypertarget{equations}{%
\section{Equations}\label{equations}}

Here is an equation.

\begin{equation} 
  f\left(k\right) = \binom{n}{k} p^k\left(1-p\right)^{n-k}
  \label{eq:binom}
\end{equation}

You may refer to using \texttt{\textbackslash{}@ref(eq:binom)}, like see Equation \eqref{eq:binom}.

\hypertarget{theorems-and-proofs}{%
\section{Theorems and proofs}\label{theorems-and-proofs}}

Labeled theorems can be referenced in text using \texttt{\textbackslash{}@ref(thm:tri)}, for example, check out this smart theorem \ref{thm:tri}.

\begin{theorem}
\protect\hypertarget{thm:tri}{}\label{thm:tri}For a right triangle, if \(c\) denotes the \emph{length} of the hypotenuse
and \(a\) and \(b\) denote the lengths of the \textbf{other} two sides, we have
\[a^2 + b^2 = c^2\]
\end{theorem}

Read more here \url{https://bookdown.org/yihui/bookdown/markdown-extensions-by-bookdown.html}.

\hypertarget{callout-blocks}{%
\section{Callout blocks}\label{callout-blocks}}

The R Markdown Cookbook provides more help on how to use custom blocks to design your own callouts: \url{https://bookdown.org/yihui/rmarkdown-cookbook/custom-blocks.html}

\hypertarget{sharing-your-book}{%
\chapter{Sharing your book}\label{sharing-your-book}}

\hypertarget{publishing}{%
\section{Publishing}\label{publishing}}

HTML books can be published online, see: \url{https://bookdown.org/yihui/bookdown/publishing.html}

\hypertarget{pages}{%
\section{404 pages}\label{pages}}

By default, users will be directed to a 404 page if they try to access a webpage that cannot be found. If you'd like to customize your 404 page instead of using the default, you may add either a \texttt{\_404.Rmd} or \texttt{\_404.md} file to your project root and use code and/or Markdown syntax.

\hypertarget{metadata-for-sharing}{%
\section{Metadata for sharing}\label{metadata-for-sharing}}

Bookdown HTML books will provide HTML metadata for social sharing on platforms like Twitter, Facebook, and LinkedIn, using information you provide in the \texttt{index.Rmd} YAML. To setup, set the \texttt{url} for your book and the path to your \texttt{cover-image} file. Your book's \texttt{title} and \texttt{description} are also used.

This \texttt{gitbook} uses the same social sharing data across all chapters in your book- all links shared will look the same.

Specify your book's source repository on GitHub using the \texttt{edit} key under the configuration options in the \texttt{\_output.yml} file, which allows users to suggest an edit by linking to a chapter's source file.

Read more about the features of this output format here:

\url{https://pkgs.rstudio.com/bookdown/reference/gitbook.html}

Or use:

\begin{Shaded}
\begin{Highlighting}[]
\NormalTok{?bookdown}\SpecialCharTok{::}\NormalTok{gitbook}
\end{Highlighting}
\end{Shaded}


  \bibliography{book.bib,packages.bib}

\end{document}
